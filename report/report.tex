\documentclass[a4paper, 10pt, twocolumn]{article}

% --- Packages ---
\usepackage[utf8]{inputenc}
\usepackage[english]{babel}
\usepackage{geometry}
\usepackage{graphicx} % Fondamentale per il logo
\usepackage{amsmath, amssymb}
\usepackage{booktabs}
\usepackage{float}
\usepackage{listings}
\usepackage{xcolor}
\usepackage{hyperref}
\usepackage{titlesec}
\usepackage{caption}

% --- Geometry Configuration ---
\geometry{top=2.5cm, bottom=2.5cm, left=1.5cm, right=1.5cm}

% --- Title Formatting ---
\titlespacing*{\section}{0pt}{1.5ex plus 1ex minus .2ex}{1ex plus .2ex}
\titlespacing*{\subsection}{0pt}{1ex plus 1ex minus .2ex}{0.5ex plus .2ex}

% --- Code Snippet Styling ---
\definecolor{codegreen}{rgb}{0,0.6,0}
\definecolor{codegray}{rgb}{0.5,0.5,0.5}
\definecolor{codepurple}{rgb}{0.58,0,0.82}
\definecolor{backcolour}{rgb}{0.95,0.95,0.92}

\lstdefinestyle{mystyle}{
    backgroundcolor=\color{backcolour},   
    commentstyle=\color{codegreen},
    keywordstyle=\color{magenta},
    numberstyle=\tiny\color{codegray},
    stringstyle=\color{codepurple},
    basicstyle=\ttfamily\footnotesize,
    breakatwhitespace=false,         
    breaklines=true,                 
    captionpos=b,                    
    keepspaces=true,                 
    numbers=left,                    
    numbersep=5pt,                  
    showspaces=false,                
    showstringspaces=false,
    showtabs=false,                  
    tabsize=2
}
\lstset{style=mystyle}

\begin{document}

\begin{titlepage}
    \onecolumn
    \centering
    
    % --- INTESTAZIONE ---
    \vspace*{1cm} % Margine superiore visivo
    
    {\scshape\LARGE University of Trento \par}
    
    \vspace{1.5cm} % Spazio fisso per separare l'istituzione dal logo
    
    % Logo: Inserito direttamente senza ambiente figure
    % width=0.35\textwidth è spesso il "golden ratio" visivo per i loghi verticali/quadrati
    \includegraphics[width=0.35\textwidth]{img/unitn_logo.png}
    
    \vfill % Spazio elastico 1
    
    % --- CONTESTO ---
    {\scshape\Large Autonomous Software Agents \par}
    \vspace{0.5cm}
    {\large Final Project Report \par}
    
    \vspace{1cm}
    
    % --- BLOCCO TITOLO ---
    % Le linee orizzontali danno struttura. 
    % Ho aggiunto un po' di "respiro" verticale dentro le linee.
    \hrule height 0.5pt
    \vspace{0.6cm}
    { \huge \bfseries Parcel Predator \par}
    \vspace{0.6cm}
    \hrule height 0.5pt
    
    \vfill % Spazio elastico 2
    
    % --- AUTORE ---
    \emph{Author:}\\[0.5cm]
    {\Large\bfseries Gianluigi Vazzoler} \\
    \large (257846)
    
    \vfill % Spazio elastico 3 (spinge la data in fondo)
    
    % --- DATA (Solo Mese e Anno) ---
    % Questo codice genera automaticamente il nome del mese in inglese + anno
    {\large 
    \ifcase\month\or January\or February\or March\or April\or May\or June\or July\or August\or September\or October\or November\or December\fi
    \ \number\year
    }
    
    \vspace*{2cm} % Margine inferiore visivo
\end{titlepage}
\twocolumn % Switch back to two columns for the report text

% --- Abstract ---
\begin{abstract}
    \emph{[TODO: Write a brief summary of the project, covering the goal, the BDI architecture approach, the multi-agent coordination strategy, and the key results obtained.]}
\end{abstract}

% --- Chapter 1 ---
\section{Introduction}
    \emph{[TODO: Introduce the context of the project.]}

    \subsection{Context and Motivation}
    \emph{[TODO: Describe the problem domain (Deliveroo.js) and why autonomous agents are needed.]}

    \subsection{Project Objectives}
    \emph{[TODO: List the main goals (e.g., maximizing score, efficient pathfinding, robust coordination).]}

% --- Chapter 2 ---
\section{System Architecture}
    \emph{[TODO: High-level overview of the system.]}

    \subsection{General Overview}
    \emph{[TODO: Briefly explain how the system is organized.]}

    \subsection{The Agent Model (BDI)}
    \emph{[TODO: Explain how the Belief-Desire-Intention model is implemented.]}

    \subsection{Environment and Perception}
    \emph{[TODO: Describe how the agent perceives the grid, parcels, and other agents.]}

% --- Chapter 3 ---
\section{Single Agent Strategy}
    \emph{[TODO: Detail the logic for a single agent.]}

    \subsection{Belief Revision}
    \emph{[TODO: How does the agent update its internal state based on new perceptions?]}

    \subsection{Option Generation and Scoring}
    \emph{[TODO: Explain the formula or logic used to rank different options (e.g., pick up vs. deliver).]}
    
    \begin{equation}
        Score = \frac{Reward}{Cost} 
    \end{equation}

    \subsection{Intention Management}
    \emph{[TODO: How does the agent select and switch intentions?]}

    \subsection{Path Planning}
    \emph{[TODO: Discuss the algorithms used (e.g., BFS, A*, etc.) for navigation.]}

% --- Chapter 4 ---
\section{Multi-Agent Coordination}
    \emph{[TODO: Detail the strategy for the team of agents.]}

    \subsection{Communication Protocol}
    \emph{[TODO: Describe the message types and the handshake process.]}

    \subsection{Negotiation and Conflict Resolution}
    \emph{[TODO: How do agents decide who goes where to avoid collisions?]}

    \subsection{Map Division (Optional)}
    \emph{[TODO: If applicable, describe how the map is split between agents.]}

% --- Chapter 5 ---
\section{PDDL Implementation}
    \emph{[TODO: Detail the integration of the external planner.]}

    \subsection{Domain Formalization}
    \emph{[TODO: Describe the predicates and actions defined in the PDDL domain.]}

    \subsection{Problem Solving Strategy}
    \emph{[TODO: When is PDDL triggered? How are problems generated?]}

    \begin{lstlisting}[language=Lisp, caption={PDDL Action Example}]
    (:action move
        :parameters (?a - agent ?from ?to - tile)
        ...
    )
    \end{lstlisting}

% --- Chapter 6 ---
\section{Implementation Details}
    \emph{[TODO: Technical details about the code.]}

    \subsection{Code Structure}
    \emph{[TODO: Overview of the files and modules.]}

    \subsection{Key Technologies}
    \emph{[TODO: Mention Node.js, libraries used, etc.]}

% --- Chapter 7 ---
\section{Experiments and Results}
    \emph{[TODO: Present the data gathered.]}

    \subsection{Experimental Setup}
    \emph{[TODO: Describe the testing environment, maps used, and opponent configuration.]}

    \subsection{Single Agent Performance}
    \emph{[TODO: Compare performance (e.g., BFS vs PDDL) in single mode.]}

    \begin{table}[htbp]
    \centering
    \begin{tabular}{lcc}
    \toprule
    \textbf{Map} & \textbf{Strategy A} & \textbf{Strategy B} \\
    \midrule
    Map 1 & 100 & 120 \\
    Map 2 & 200 & 180 \\
    \bottomrule
    \end{tabular}
    \caption{Score comparison}
    \label{tab:results}
    \end{table}

    \subsection{Multi-Agent Performance}
    \emph{[TODO: Analyze the efficiency of the team coordination.]}

    \begin{figure}[htbp]
        \centering
        \framebox[\linewidth]{\rule{0pt}{3cm}Insert Chart Here}
        \caption{Multi-agent performance graph}
        \label{fig:chart}
    \end{figure}

% --- Chapter 8 ---
\section{Discussion and Conclusion}
    \emph{[TODO: Discuss the strengths and weaknesses of the solution. Conclude with future improvements.]}

% --- Bibliography ---
\begin{thebibliography}{9}
    \bibitem{slides}
    P. Giorgini, M. Robol, \emph{Autonomous Software Agents Course Slides}, UniTrento, 2025.
\end{thebibliography}

\end{document}